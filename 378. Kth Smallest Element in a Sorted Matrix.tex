\documentclass{article}
\usepackage{amsmath,amsfonts,amsthm,amssymb}
\usepackage{setspace}
\usepackage{fancyhdr}
\usepackage{lastpage}
\usepackage{extramarks}
\usepackage{chngpage}
\usepackage{soul,color}
\usepackage{graphicx,float,wrapfig}
\usepackage[colorlinks,linkcolor=blue]{hyperref}
\usepackage{parskip}
\setlength{\parindent}{0cm}

% In case you need to adjust margins:
\topmargin=-0.45in      %
\evensidemargin=0in     %
\oddsidemargin=0in      %
\textwidth=6.5in        %
\textheight=9.0in       %
\headsep=0.25in         %

\newcommand{\Answer}{\ \\\textbf{Answer:} }
\newcommand{\Proof}{\ \\\textbf{Proof:} }
\newcommand{\Acknowledgements}[1]{\ \\{\bf Acknowledgements:} #1}
\newcommand{\Infer}{\Longrightarrow}
\newcommand{\ud}{\mathrm{d}}
\newcommand{\Reduce}{\Longleftarrow}
\newcommand{\Endproof}{\hfill $\Box$ \\}
\newcommand{\T}{\mathrm{T}}
\newcommand{\E}{\mathbb{E}}
\newcommand{\Real}{\mathbb{R}}
\newcommand*\circled[1]{\tikz[baseline=(char.base)]{\node[shape=circle,draw,inner sep=2pt] (char) {#1};}}


\begin{document}

%%%%%%%%%%%%%%%%%%%%%%%%%%%%%%%%%%%%%%%%%%%%%%%%%%%%%%%%%%%%%
% Begin edit from here

1. Given a value $x$, we can know the rank of $x$ in the matrix in $O(n)$ time, by monotone pointers. Using binary search (on $n^2$ elements), the running time is $O(n\log U)$ or $O(n\log n)$.\\
2. $O(\min(k,m))$ if it's a $n\times m$ matrix and $n\leq m$.\\ \url{https://chaoxuprime.com/posts/2014-04-02-selection-in-a-sorted-matrix.html}\\

for small $k$:\\
3. put the first element in each row into a heap, whenever we pop, add the next element in the corresponding row. $O(k\log n)$.\\
4. $O(\sqrt k\log k)$. The crucial observation is that there are $xy$ elements smaller than entry $(x,y)$, so the possible region for the solution can be divided into $2\sqrt k$ rows and columns (i.e. row/columns $1,\dots,\sqrt k$). use results on selection in union of sorted arrays.\\
see reference for 4. median of two sorted arrays.\\

note. is \cite{frederickson1993optimal} useful to get $O(k)$?\\


\bibliographystyle{plain}
\bibliography{references}


%\Acknowledgements{Thank XXX XX 2007010002 for the discussion about ...}
%\begin{thebibliography}{}
%\bibitem{}{Discrete Mathematics: Elementary and Beyond}
%\end{thebibliography}
% End edit to here
%%%%%%%%%%%%%%%%%%%%%%%%%%%%%%%%%%%%%%%%%%%%%%%%%%%%%%%%%%%%%

\end{document}

%%%%%%%%%%%%%%%%%%%%%%%%%%%%%%%%%%%%%%%%%%%%%%%%%%%%%%%%%%%%%
